\chapter{Введение}

\section{Цель}

Изучение и освоение способов организации вычислительных кластеров на основе существующей сети рабочих станций.

\section{Используемые технологии}
Для построения кластера были использованы следующий комплекс программного обеспечения:

\begin{enumerate}
  \item KVM --- средство виртуализации.
  \item PelicanHPC GNU Linux --- операционная система на виртуальных компьютерх, формирующих кластер.
\end{enumerate}

\section{Краткий обзор и анализ существующих технологий}

Sierra (ATS-2) — суперкомпьютер, созданный для Ливерморской национальной лаборатории им. Лоуренса. Используется Национальным управлением по ядерной безопасности США (NNSA) в качестве второй системы Advanced Technology System. Используется для решения задач, связанных с управлением ядерным арсеналом США, помогая обеспечить безопасность, надежность и эффективность ядерного оружия Соединенных Штатов. Теоретическая производительность Sierra оценивается в 125 ПФлопс, практически достигнута на тесте производительности LINPACK 94,6 ПФлопс.

Sierra похож по архитектуре на суперкомпьютер Summit, установленный в Национальной лаборатории Ок-Риджа. Система Sierra использует центральные процессоры IBM POWER9 в сочетании с графическими процессорами Nvidia Tesla V100. Узлами в Sierra являются серверы Witherspoon S922LC OpenPOWER с двумя центральными процессорами и четырьмя графическими процессорами на узел. Эти узлы связаны между собой при помощи сети EDR InfiniBand.

Summit (OLCF-4) — суперкомпьютер Ок-Риджской национальной лаборатории вычислительной мощностью 122,3 ПФлопс, продемонстрированной на тесте HPL; являлся самым высокопроизводительным компьютером в период с июня 2018 года по июнь 2020 года в открытом рейтинге суперкомпьютеров Top500.

Контракт на \$325 млн Министерства энергетики США на построение суперкомпьютера в 2014 году получили IBM (серверные узлы), Mellanox (межсоединение) и Nvidia (графические ускорители). Система введена в эксплуатацию в июне 2018 года, заменив суперкомпьютер Titan (OLCF-3).

Комплекс занимает площадь около 520 м² и состоит из 4608 серверных узлов IBM Power Systems AC922, в общей сложности суперкомпьютер оснащён 9216 22-ядерными процессорами IBM POWER9 и 27 648 графическими процессорами NVIDIA Tesla V100. Каждый узел содержит более 500 ГБ когерентной памяти (High Bandwidth Memory и DDR4 SDRAM), которая адресуется всеми CPU и GPU, плюс 800 ГБ энергонезависимой памяти, которая может использоваться как пакетный буфер или дополнительная память. Процессоры и видеокарты подключаются с использованием шины NVLink, что позволяет использовать гетерогенную вычислительную модель.

В подсистеме охлаждения циркулирует 15 150 литров очищенной воды; потребляемая мощность системы в целом — 15 МВт (что сравнивается с электропотреблением 8100 среднестатистических жилых домов на одно семейство в США). 

22 июня 2020 года опубликован 55-й выпуск рейтинга топ-500 самых высокопроизводительных систем мира. В списке суперкомпьютеров мира произошла смена лидера. Вместо американского Summit (IBM) на первом месте теперь японский кластер Fugaku на процессорах ARM A64FX 48C 2.2 GHz. Заявленная пиковая производительность Fugaku составляет 513.8 петафлопс, что в 2,5 раза больше, чем у суперкомпьютера Summit на IBM Power9.

Суперкомпьютер Fugaku был спроектирован и собран в японском институте физико-химических исследований RIKEN. Его разработка была инициирована Министерством образования, культуры, спорта, науки и технологий Японии в 2014 году. Инженерам и специалистам по высокопроизводительным системам была поставлена цель — создать флагманский суперкомпьютер следующего поколения (эксафлопсный суперкомпьютер).

Суперкомпьютера Fugaku состоит из 158 976 узлов на базе SoC Fujitsu A64FX (в каждом из них на борту 48-ядерный CPU Armv8.2-A SVE (512 bit SIMD) с тактовой частотой 2.2 ГГц). В сумме кластер Fugaku насчитывает более 7 млн процессорных ядер, около 5 ПБ ОЗУ и 150 ПБ общего хранилища на базе ФС Lustre. В качестве операционной системы в Fugaku используется Red Hat Enterprise Linux. 

SoC Fujitsu A64FX уникален тем, что он является первым процессором, в котором объединена поддержка памяти HBM2 и векторных расширений архитектуры ARM (Scalable Vector Extensions или SVE). Использование HBM2 обеспечивает A64FX теоретическую пропускную способность памяти более 1 ТБ/с, а поддержка Arm SVE повышает производительность в задачах искусственного интеллекта и аналитики. Японские инженеры уже несколько лет прорабатывают решения для применения этих особенности A64FX как для исследований в области ядерной физики так и в других отраслях науки и промышленности, использую преимущества систем на базе Arm — высокая степень распараллеливания, низкое энергопотребление и высокая надежность.

