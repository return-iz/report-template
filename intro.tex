\chapter{Введение}

\section{Цель работы}

\begin{itemize}
	\item получить базовые навыки работы в среде R;
	\item изучить средства R для проведения первичного разведочного анализа данных (методы визуализации, описательной статистики, корреляционного анализа данных) на примере решения конкретной задачи ИАД (интеллектуального анализа данных).
\end{itemize}


\section{Задание для варианта №5}
Изучаются показатели работы программистов крупной организации. Рассматриваются следующие показатели (признаки) для каждого программиста:

\begin{itemize}
	\item пол (1-м, 2-ж);
	\item возраст;
	\item стаж работы;
	\item процент разработок, выполненных в срок в рамках бюджета с требуемым функционалом (за год);
	\item количество ошибок, выявленных пользователем (за год);
	\item стаж работы по специальности в данной организации;
	\item степень удовлетворенности заказчика;
	\item качество документирования (1 – низкое, 2 – среднее, 3 – выше среднего, 4- высокое).
\end{itemize}

Необходимо провести предварительный разведочный анализ данных с целью описания характера распределения данных, выявления структуры взаимосвязей между показателями.

Программисты разбиты на две группы в зависимости от стажа работы:

\begin{itemize}
	\item 1 группа - стаж менее 4
	\item 2 группа - стаж более 4
\end{itemize}