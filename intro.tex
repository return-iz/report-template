\chapter{Введение}

\section{Цель}

Изучение и освоение способов организации вычислительных кластеров на основе существующей сети рабочих станций.

\section{Используемые технологии}
Для построения кластера были использованы следующий комплекс программного обеспечения:

\begin{enumerate}
  \item KVM --- средство виртуализации.
  \item PelicanHPC GNU Linux --- операционная система на виртуальных компьютерх, формирующих кластер.
\end{enumerate}

\section{Краткий обзор и анализ существующих технологий}

22 июня 2020 года опубликован 55-й выпуск рейтинга топ-500 самых высокопроизводительных систем мира. В списке суперкомпьютеров мира произошла смена лидера. Вместо американского Summit (IBM) на первом месте теперь японский кластер Fugaku на процессорах ARM A64FX 48C 2.2 GHz. Заявленная пиковая производительность Fugaku составляет 513.8 петафлопс, что в 2,5 раза больше, чем у суперкомпьютера Summit на IBM Power9.

Суперкомпьютер Fugaku был спроектирован и собран в японском институте физико-химических исследований RIKEN. Его разработка была инициирована Министерством образования, культуры, спорта, науки и технологий Японии в 2014 году. Инженерам и специалистам по высокопроизводительным системам была поставлена цель — создать флагманский суперкомпьютер следующего поколения (эксафлопсный суперкомпьютер). Его официальный запуск в промышленную эксплуатацию будет в 2021 году, к этому времени Fugaku будет еще модернизирован.

В настоящее время специалисты института RIKEN занимаются усовершенствованием и доработкой систем и элементов Fugaku.

Суперкомпьютера Fugaku состоит из 158 976 узлов на базе SoC Fujitsu A64FX (в каждом из них на борту 48-ядерный CPU Armv8.2-A SVE (512 bit SIMD) с тактовой частотой 2.2 ГГц). В сумме кластер Fugaku насчитывает более 7 млн процессорных ядер, около 5 ПБ ОЗУ и 150 ПБ общего хранилища на базе ФС Lustre. В качестве операционной системы в Fugaku используется Red Hat Enterprise Linux. 

SoC Fujitsu A64FX уникален тем, что он является первым процессором, в котором объединена поддержка памяти HBM2 и векторных расширений архитектуры Arm (Scalable Vector Extensions или SVE). Использование HBM2 обеспечивает A64FX теоретическую пропускную способность памяти более 1 ТБ/с, а поддержка Arm SVE повышает производительность в задачах искусственного интеллекта и аналитики. Японские инженеры уже несколько лет прорабатывают решения для применения этих особенности A64FX как для исследований в области ядерной физики так и в других отраслях науки и промышленности, использую преимущества систем на базе Arm — высокая степень распараллеливания, низкое энергопотребление и высокая надежность.

Кстати, у проекта Fujitsu A64FX есть свой открытый репозиторий в GitHub, где размещена документация и примеры для разработчиков.

Рейтинг топ-500 суперкомпьютеров мира составляется с 1993 года два раза в год (в июне и ноябре) специалистами и учеными из США и Германии. Минимальный порог для попадания в него в этом году достиг 1,702 петафлопс. В новой версии рейтинга топ-500 суперкомпьютеров мира Китаю принадлежат 226 систем, 114 относятся к США и 29 расположены в Японии.

Количество российских суперкомпьютеров в рейтинге топ-500 самых высокопроизводительных систем мира уменьшилось. Теперь там только две отечественные высокопроизводительные системы — на 36 месте SberCloud (пиковая производительность 8,789 петафлопс) и на 131 месте Lomonosov 2 (пиковая производительность 4,947 петафлопс). В этом году суперкомпьютер ФГБУ ГВЦ Росгидромета (пиковая производительность 1,3 петафлопс) выбыл из нового списка топ-500 самых высокопроизводительных систем мира.

Ранее в конце марта 2020 года был обновлен рейтинг 50 самых производительных суперкомпьютеров СНГ (редакция №32 от 31.03.2020), в котором достаточно долгое время не появляются новые высокопроизводительные системы, исключение — суперкомпьютер Christofari, разработанный облачной платформой «Сбербанка» SberCloud. Все 50 суперкомпьютеров СНГ в качестве основных процессоров используют чипы Intel и AMD, также среди них 28 гибридных систем, использующих для вычислений ускорители.

На данный момент суперкомпьютеров на базе отечественной архитектуры «Эльбрус» в списке топ-50 суперкомпьютеров СНГ нет, хотя ранее в концерне «Автоматика» госкорпорации Ростех анонсировали появление первого суперкомпьютера на 8-ядерных микропроцессорах «Эльбрус-8С», который предназначен для выполнения высокопроизводительных вычислений, обработки больших объемов данных и решения задач с повышенными требованиями к информационной безопасности.
