\section{Перечень и основные характеристики программного обеспечения}

Несмотря на то, что средства MPI сами по себе позволяют осуществлять запуск параллельных задач, обычно для этих целей используются различные менеджеры ресурсов.

Можно назвать следующее программное обеспечение для управления кластером:

\begin{itemize}
	\item Torque,
	\item Maui,
	\item Corosync.
\end{itemize}

\paragraph{Torque}
Менеджер распределённых ресурсов для вычислительных кластеров из машин под управлением Linux и других Unix-подобных операционных систем, одна из современных версий Portable Batch System (PBS). Распространяется под свободной лицензией OpenPBS Software License. TORQUE разрабатывается и поддерживается сообществом на базе проекта OpenPBS. Для менеджера существует более 1200 патчей и расширений, написанных крупнейшими организациями и лабораториями, среди которых US DOE, USC, PNLL и др., это позволяет достичь высокой степени масштабируемости и отказоустойчивости менеджера как системы.

\paragraph{Maui}
Планировщик заданий в параллельных и распределенных вычислительных системах (кластерах). Как правило, используется совместно с менеджером распределенных ресурсов TORQUE.
Maui позволяет выбирать различные политики планирования, поддерживает динамическое изменение приоритетов, исключения. Все это улучшает управляемость и эффективность машин, начиная от простых кластеров до суперкомпьютеров.

\paragraph{Corosync}
Проект с открытым исходным кодом, реализующий систему группового общения для отказоустойчивых кластеров. Является развитием проекта OpenAIS и опубликован в соответствии с модифицированной лицензией BSD. Программное обеспечение создано как исполняемые бинарные файлы, использующие клиент-серверную модель взаимодействия между библиотеками и сервисными инструментами. Модули, называемые сервисными инструментами, загружаются в Corosync и используют сервисы, предоставляемые внутренним API Corosync.

В ходе выполнения лабораторной работы будет произведено развертывание системы Torque.