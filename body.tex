\chapter{ОСНОВНАЯ ЧАСТЬ}

\section{Разворачивание контейнера PostgreSQL}
Первое, что необходимо сделать, --- это развернуть базу данных. Есть несколько способов реализации этого.

Первый заключается в том, чтобы установить сервер базы данных непосредственно <<на железе>>. Подход этот достаточно простой, но во-первых, он интегрирует в систему различные файлы, которые может быть не легко удалить, если понадобиться, а во-вторых данный вариант не масштабируется.

Второй --- воспользоваться контейнеризацией. Этот подход избавлен от все недостатков предыдущего. Воспользуемся Docker и Docker Compose для развертывания сервера БД PostgreSQL. Создадим файл docker-compose.yml и запишем в нем следующее:

\begin{minted}{yaml}
version: "3"

services:
  db:
    image: postgres
    restart: always
    environment:
      POSTGRES_PASSWORD: postgres
      POSTGRES_DB: laba3
    ports:
      - 5432:5432
\end{minted}

Описав правила развертывания выполним их с помощью команды:

\begin{minted}{bash}
$ docker-compose up
\end{minted}

После этого сервер базы данных доступен на локальной машине через порт 5432. Для управления БД через WEB-интерфейс можно добавить в инфраструктуру еще один контейнер, который построен из образа adminer. В данной лабораторной работе в этом нет необходимости.

\section{Создание модели и ее контекста}
ASP.NET Core пропагандирует подход Code First. Суть этого подхода заключается в том, что сначала в коде описываются сущности базы данных и их связи, а затем по этим описаниям сам фреймворк строит структуру БД. Этот подход является современным и используется сейчас повсеместно в разных фреймворках. Тоже воспользуемся им и опишем модель предмета кафедры в файле Models/Discipline.cs:

\begin{minted}{csharp}
namespace Labs.Models
{
  public class Discipline
  {
    public int Id { get; set; }
    public string Name { get; set; }
  }
}
\end{minted}

Для простоты модель будет содержать только ID и название предмета. В действительности можно выделить значительно большое полей, которые описывают дисциплину кафедры.

Фреймворк не работает с моделями на прямую. Модели нужны только как отображение таблиц БД в виде классов. Само же выполнение команд БД выполняется через сущности, которые называются контекстом. Создадим контекст модели в файле Models/DepartmentContext.cs:

\begin{minted}{csharp}
namespace Labs
{
  public class DepartmentContext : DbContext
  {
    public DbSet<Discipline> dp { get; set; }
    
    public DepartmentContext()
    {
      Database.EnsureCreated();
    }

    protected override void 
    OnConfiguring(DbContextOptionsBuilder optionsBuilder)
    {
      optionsBuilder.UseNpgsql(
                               "Host=localhost;" +
                               "Port=5432;" +
                               "Database=laba3;" + 
                               "Username=postgres;" + 
                               "Password=postgres");
    }
  }
}
\end{minted}

В конструкторе контекста проверяется существуют ли описываемые сущности. Если чего-то не хватает, то это досоздается. В переопределенной функции задаются действия при конфигурации. Конкретно в этом случае задается строка подключения к базе данных. Также в контексте создается свойство-множество, которое содержит дисциплины кафедры.

\section{Создание контроллера}
Создадим контроллер для данной лабораторной работы. Он будет содержать статическое свойство контекста модели, а также три метода: для главной страницы, для удаления и для добавления записей в базу данных. Создадим контроллер по пути Controllers/Laba3Controller.cs:

\begin{minted}{csharp}
namespace Labs.Controllers
{
  public class Laba3 : Controller
  {
    public static DepartmentContext db = new DepartmentContext();
		
    public IActionResult Index()
    {
      return View(db.dp.ToList());
    }
		
    public IActionResult Delete(int id)
    {
      Console.WriteLine(id);
      var dp = db.dp.Where(obj => obj.Id == id).FirstOrDefault();
      db.dp.Remove(dp);
      db.SaveChanges();
      return Redirect("~/Laba3/Index");
    }
		
    [HttpPost]
    public IActionResult Add(string name)
    {
      var dp = new Discipline();
      dp.Name = name;
      db.dp.Add(dp);
      db.SaveChanges();
      return Redirect("~/Laba3/Index");
    }
  }
}
\end{minted}

Для созданного контроллера будет достаточно одного представления, поскольку только один метод возвращает его -- Index(). Методы Add() и Delete() не возвращает представление, они лишь выполняют утилитарные функции по добавлению и удалению записей в/из базы данных.

\section{Создание представления}
Представление должно находиться по пути Views/Laba3/Index.cshtml и содержать следующий код:

\begin{minted}{html}
@model List<Labs.Models.Discipline>

<form action="~/Laba3/Add" method="post">
  <label for="name">Название предмета</label>
  <input type="text" name="name">
  <input type="submit" value="Добавить">
</form>

<table border="1" >
  <tr>
    <td>ID</td>
    <td>Название</td>
    <td>Действие</td>
  </tr>
  @foreach (var dp in Model)
  {
    <tr>
      <td>@dp.Id</td>
      <td>@dp.Name</td>
      <td><a href="~/Laba3/Delete?id=@dp.Id">Удалить</a></td>
    </tr>
  }
</table>
\end{minted}

Представление принимает модель в виде списка дисциплин кафедры. Затем создается форма для добавления новой дисциплины в список. Ниже генерируется таблица с рамкой. Таблица содержит три столбца: ID, название, действие. ID и название берутся при формировании строки из свойств модели. Действие задается в виде ссылки на метод Delete() контроллера. В качестве параметра запроса используется ID текущей обрабатываемой записи.