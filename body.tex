\chapter{ОСНОВНАЯ ЧАСТЬ}

Модель данных аналогична предыдущей лабораторной работе. Однако представление и контроллер будут отличаться.

\section{Создание представления}

Представление должно поддерживать отображение всех элементов базы данных, а также иметь кнопки для сохранения, очистки и восстановления БД.

\begin{minted}{html}
@using pois.Models

@model IEnumerable<Discipline>

<table border="1">
  <tr>
    <td>ID</td>
    <td>Название</td>
  </tr>
  @foreach (var d in Model)
  {
    <tr>
      <td>@d.Id</td>
      <td>@d.Name</td>
    </tr>
  }
</table>

<div>
  <form action="~/Home/Save" method="get">
    <input type="submit" value="Сохранить данные">
  </form>

  <form action="~/Home/Clear" method="get">
    <input type="submit" value="Очистить данные">
  </form>

  <form action="~/Home/Load" method="get">
    <input type="submit" value="Восстановить данные">
  </form>
</div>
\end{minted}

Нажатие на кнопки вызывает соответствующие методы из контроллера.

\section{Создание контроллера}

Контроллер должен уметь возвращать представление, а также сохранять БД в JSON-файл, очищать ее и восстанавливать. 

\begin{minted}{csharp}
namespace pois.Controllers
{
  public class HomeController : Controller
  {
    public static DepContext context = new DepContext();

    public IActionResult Index()
    {
      return View(context.Disciplines.ToList());
    }

    public IActionResult Save()
    {
      using (var fs = new StreamWriter(@"wwwroot/data.json", false ))
      {
        foreach (var d in context.Disciplines.ToList())
        {
          var jsonText = JsonSerializer.Serialize<Discipline>(d);
          fs.WriteLine(jsonText);
        }
      }
      return Redirect("~/Home/Index");
    }

    public IActionResult Clear()
    {
      context.Disciplines.RemoveRange(context.Disciplines);
      context.SaveChanges();
      return Redirect("~/Home/Index");
    }

    public IActionResult Load()
    {
      using (var fs = new StreamReader(@"wwwroot/data.json"))
      {
        context.Disciplines.RemoveRange(context.Disciplines);
        context.SaveChanges();
        string line;
        while ((line = fs.ReadLine()) != null)
        {
          var d = JsonSerializer.Deserialize<Discipline>(line);
          context.Disciplines.Add(d);
          context.SaveChanges();
        }
      }
      return Redirect("~/Home/Index");
    }
  }
}
\end{minted}

Сохранение, очистку и восстановление выполняют соответственно методы Save(), Clear() и Load(). Данные сериализуются в директорию общедоступных файлов в файл с названием data.json. Каждый из этих методов перенаправляет на главную страницу послы выполнения нужных действий.